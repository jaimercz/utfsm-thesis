%!TEX root = ../memoria.tex

\chapter{¿Cómo usar esta Plantilla?}

Para instrucciones sobre el uso y configuración de esta plantilla, por favor, revise el documento \inlinecode{memoria.tex} con un editor de texto o editor de \LaTeX{}.

Toda configuración general se encuentra en el documento maestro (\inlinecode{memoria.tex}). Ahí podrá cambiar los parámetros de la portada y los documentos a incluir. Por ejemplo, si necesita más capítulos, simplemente puede agregarlos creando el archivo (ej. \inlinecode{/includes/capitulo04.tex})  e incorporando la siguiente línea en el documento maestro:

\inlinecode{\\input\{includes/capitulo04\}}

\begin{Verbatim}[frame=lines, label=\inlinecode{memoria.tex} (extracto)
				, fontsize=\footnotesize
				, baselinestretch=1
				, formatcom=\color{gray}]
%%%%%%%%%%%%%%%%%%%%%%%%%%%%%%%%%%%%%
%	Cuerpo Principal (Main Matter)
%%%%%%%%%%%%%%%%%%%%%%%%%%%%%%%%%%%%%
\mainmatter
\pagestyle{fancy}

%!TEX root = ../memoria.tex

\chapter{¿Cómo usar esta Plantilla?}

\section{Obtener el código fuente \LaTeX}

Primero, por supuesto, obtener la plantilla y los archivos de apoyo desde GitHub (\url{https://github.com}):


\inlinecode{git clone https://github.com/jaimercz/utfsm-thesis}

\section{Configuración}

La configuración básica (nombre del autor, comisión evaluadora, fecha, grado y título de la memoria o tesis) está en el archivo \inlinecode{config.tex}. Modifique ahí los parámetros básicos de este documento (que afectan la portada y los meta-datos PDF).

\section{Compilar (primera vez)}

Abra el documento \inlinecode{memoria.tex} con un editor de texto o editor de \LaTeX{} de su preferencia.

Proceda con la compilación:

\begin{Verbatim}[frame=lines, label=Consola (Shell) o Línea de comandos
, fontsize=\footnotesize
, baselinestretch=1
, formatcom=\color{gray}]
    $ pdflatex memoria.tex
    $ biber memoria
    $ pdflatex memoria.tex
    $ pdflatex memoria.tex
\end{Verbatim}

Si hay errores, lo más probable es que le falte alguno de las paquetes necesarios que ocupa esta plantilla.

Esta version hace uso de \inlinecode{biber} en lugar de \inlinecode{natbib / bibtex}. Natbib es del año 1988, y el manejo de documentos digitales modernos no estaba contemplado entonces.

\section{Modificación de contenidos}


Abrir el documento maestro (\inlinecode{memoria.tex}) y modificar o incluir los documentos que componen su memoria.

Por ejemplo, para incorporar un nuevo capítulo, simplemente puede agregarlo incorporando la siguiente línea en el documento maestro:

\inlinecode{\\input\{includes/capitulo04\}}

\begin{Verbatim}[frame=lines, label=\inlinecode{memoria.tex} (extracto)
				, fontsize=\footnotesize
				, baselinestretch=1
				, formatcom=\color{gray}]
% \input{includes/capitulo04}
% \input{includes/capitulo05}
%...              % Agregar aquí más capítulos
\end{Verbatim}

\section{Codificación de caracteres}

Todos los archivos \inlinecode{*.tex} de esta plantilla han sido preparados ocupando la codificación de caracteres por defecto \emph{unicode} (UTF-8). Windows (y algunas versiones de OSX) ocupan otro tipo de codificación (ej. \emph{Windows-1252} o \emph{Mac Roman}).

Si deseas ocupar esta plantilla y encuentras problemas con los caracteres acentuados, entonces puedes optar por una de estas tres alternativas:
\begin{enumerate}[(i)]
    \item cambiar tu editor (TexMaker, TexStudio, TexShop, etc.) para que ocupe UTF-8 como codificación de caracteres por defecto; o
    \item cambiar la codificación de cada documento \inlinecode{*.tex} para que ocupe la codificación nativa de tu sistema operativo; y, modifica la configuración (\inlinecode{config.tex}) dice:
    
    OSX, *nix: \inlinecode{\\usepackage[utf8x]\{inputenc\}}

    Windows: \inlinecode{\\usepackage[latin1]\{inputenc\}}

    Overleaf: \inlinecode{\\usepackage[utf8]\{inputenc\}} (\url{https://overleaf.com})

    \item escribir todo ocupando caracteres pre-acentuados (ej. \inlinecode{\\'a} en lugar de á).
\end{enumerate}

\vspace{10mm}
\begin{framed}
    \textbf{Recuerda:} Mezclar documentos de distintas codificaciones puede generarte muchos problemas al momento de compilar.  
\end{framed}

\newpage
\section{Requisitos}
Los paquetes que se ocupan y son indispensables para la generación este documento están contenidos en el documento de clase \inlinecode{thesis_utfsm.cls}.

Para que funcione correctamente se requiere tener instaladas (como mínimo) las siguientes extensiones \LaTeX{}:
\begin{Verbatim}[frame=lines, label=Paquetes requeridos por \inlinecode{thesis_utfsm.sty}
				, fontsize=\footnotesize
				, baselinestretch=1
				, formatcom=\color{gray}]
geometry    % Márgenes y tamaño de páginas
biblatex    % Bibliografía
fontenc     % Codificación de Caracteres
inputenc    % Métodos de entrada (acentos)
fancyhdr    % Encabezados 'Fancy'
chngcntr    % Formatos de Pie de Página
booktabs    % Tablas
tabularx    % Tablas
multirow    % Tablas con multi-columnas / multi-filas
array       % Matrices
float       % Imágenes Flotantes
textcomp    % Símbolos de uso común
endnotes    % Notas finales del documento
paralist    % Mejores Listados
listings    % Mejores Listados
framed      % Marcos
fancybox    % Marcos 'Fancy'
verbatim    % Código Fuente
fancyvrb    % Código Fuente 'Fancy'
wrapfig     % Figuras flotantes
xcolor      % Colores personalizados
graphix     % Mejor inclusión de figuras
subfig      % Figuras con múltiples leyendas
tikz        % Diagramas vectoriales
caption     % Mejores leyendas para figuras y tablas
tocbibind   % Bibliografía en la Tabla de Contenidos
rotating    % Rotación de Tablas
asmmath     % Notación ciéntifica / matemática
asmsymb     % Símbolos matemáticos y letras griegas
txfonts     % Times New Roman (para sistemas distintos de Windows)
microtype   % Mejoras subliminales en el uso de fuentes
parskip     % Separación entre párrafos
\end{Verbatim}

La mayoría de las distribuciones \LaTeX{} traen estos paquetes por defecto, sin embargo, en Windows es posible que deba instalar algunos de ellos si ha instalado el archivo básico de MikTeX.


\newpage
%%%%%
\section{Diagramación}
Este documento fue realizado usando \LaTeX{} (\citet{latex:whatis}), aunque puede fácilmente ser exportado a LyX (\citeauthor{lyx}). Para ver como transformarlo a Lyx, puede revisar el Wiki (\citeauthor{wikilyx}).

Usted necesitará un compilador de \LaTeX. Los más comúnmente ocupados son \citeauthor{miktex} (Windows) y \citeauthor{mactex} (Apple); Sistemas *nix (incluyendo linux) traen \TeX{} por defecto.

Para una referencia completa sobre \LaTeX{}, recomendamos el libro de \cite{Lamport94}; aunque para solucionar problemas específicos, su mejor aliado es Internet.

% Other Author (Included only in Bibliography)
También puede revisar \citet{Roberts05}, \citet{Oetiker06}, y \citet{Mittelbach04}.

\subsection{Figuras}
La siguiente es una figura basada en el archivo \inlinecode{figures/logoind.png}. En este caso, la descripción de la figura va en la parte inferior (ver \autoref{fig:logoind}).

% Inclusión de Figuras
\begin{figure}[ht!]
\centering
\includegraphics[width=.3\textwidth]{figures/logoind.png}
\caption[Logotipo Departamento de Industrias]{Logotipo Departamento de Industrias\\
{\scriptsize (Fuente: Departamento de Industrias)}}
\label{fig:logoind}
\end{figure}

La forma de incorporar la \autoref{fig:logoind} se muestra a continuación:


\begin{Verbatim}[frame=lines, label=Incorporar \autoref{fig:logoind}
				, fontsize=\footnotesize, numbers=left
				, baselinestretch=1
				, formatcom=\color{gray}]
\begin{figure}[h]
\centering
\includegraphics[width=.4\textwidth]{figures/logoind.png}
\caption[Logotipo Departamento de Industrias]{Logotipo Departamento de Industrias\\
{\scriptsize (Fuente: Departamento de Industrias)}}
\label{fig:logoind-demo}
\end{figure}
\end{Verbatim}

Otra forma de incorporar figuras es mediante un \inlinecode{float}. En este caso, la figura es incorporada como una imagen ``flotante'' a un costado del texto  (ver Figura \autoref{fig:logousm}).

\begin{wrapfigure}{o}{.3\textwidth}
    \vspace{-20pt}
    \begin{spacing}{1}
        \begin{center}
            \includegraphics[width=.25\columnwidth]{figures/logousm.png}
            \vspace{-10pt}
            \caption{Logotipo USM (Float)}
            \label{fig:logousm}
        \end{center}
    \end{spacing}
    \vspace{-10pt}
\end{wrapfigure}

Lorem ipsum dolor sit amet, consectetuer adipiscing elit. Ut purus elit, vestibulum ut, placerat ac, adipiscing vitae, felis. Curabitur dictum gravida mauris. Nam arcu libero, nonummy eget, consectetuer id, vulputate a, magna. Donec vehicula augue eu neque. Pellentesque habitant morbi tristique senectus et netus et malesuada fames ac turpis egestas. Mauris ut leo. Cras viverra metus rhoncus sem. Nulla et lectus vestibulum urna fringilla ultrices. Phasellus eu tellus sit amet tortor gravida placerat. Integer sapien est, iaculis in, pretium quis, viverra ac, nunc. Praesent eget sem vel leo ultrices bibendum. Aenean faucibus. Morbi dolor nulla, malesuada eu, pulvinar at, mollis ac, nulla. Curabitur auctor semper nulla. Donec varius orci eget risus. Duis nibh mi, congue eu, accumsan eleifend, sagittis quis, diam. Duis eget orci sit amet orci dignissim rutrum.



\begin{Verbatim}[frame=lines, label=\autoref{fig:logousm}
				, fontsize=\footnotesize, numbers=left
				, baselinestretch=1
				, formatcom=\color{gray}]
\begin{wrapfigure}{o}{.4\textwidth}
    \vspace{-20pt}
    \begin{spacing}{1}
        \begin{center}
            \includegraphics[width=.35\columnwidth]{figures/logousm.png}
            \vspace{-10pt}
            \caption{Logotipo USM (Float)}
            \label{fig:logousm-demo}
        \end{center}
    \end{spacing}
    \vspace{-10pt}
\end{wrapfigure}
\end{Verbatim}


\subsection{Tablas}

La siguiente es una tabla o cuadro básica (ver \autoref{tbl:temperaturas}). Notar las referencias cruzadas y el título de la tabla en la parte superior.

\begin{table}[h!]
    \caption[Ejemplo: Tabla de Temperaturas]{Tabla de Temperaturas}
    \label{tbl:temperaturas}
    \begin{tabularx}{\linewidth}{@{} l  c  c  X @{}}
        \toprule
        \textbf{\textsc{Day}} &  \textbf{\textsc{Min Temp}} 
        		& \textbf{\textsc{Max Temp}} & \textbf{\textsc{Summary}}\\
    	  \hline\hline
        Monday & 11C & 22C & A clear day with lots of sunshine.
        However, the strong breeze will bring down the temperatures. \\ \hline
        Tuesday & 9C & 19C & Cloudy with rain, across many northern regions. Clear spells
        across most of Scotland and Northern Ireland,
        but rain reaching the far northwest. \\ \hline
        Wednesday & 10C & 21C & Rain will still linger for the morning.
        Conditions will improve by early afternoon and continue
        throughout the evening. \\
        \bottomrule
    \end{tabularx}
\end{table}

\begin{Verbatim}[frame=lines, label=\autoref{tbl:temperaturas} Alternative
				, fontsize=\footnotesize, numbers=left
				, baselinestretch=1
				, formatcom=\color{gray}]

\begin{table}[h!]
    \caption[Ejemplo: Tabla de Temperaturas]{Tabla de Temperaturas}
    \label{tbl:temperaturas-demo}
    \begin{tabularx}{\linewidth}{@{} l  c  c  X @{}}
        \toprule
        \textbf{\textsc{Day}} &  \textbf{\textsc{Min Temp}} 
        & \textbf{\textsc{Max Temp}} & \textbf{\textsc{Summary}}\\
        \hline\hline
        Monday & 11C & 22C & A clear day with lots of sunshine.
        However, the strong breeze will bring down the temperatures. \\ \hline
        Tuesday & 9C & 19C & Cloudy with rain, across many northern regions. Clear spells
        across most of Scotland and Northern Ireland,
        but rain reaching the far northwest. \\ \hline
        Wednesday & 10C & 21C & Rain will still linger for the morning.
        Conditions will improve by early afternoon and continue
        throughout the evening. \\
        \bottomrule
    \end{tabularx}
\end{table}
\end{Verbatim}



\subsubsection{Rotación de Tablas}
En caso de tener tablas muy grandes, o si necesita una tabla rotada, puedes ocupar \inlinecode{sidewaystable} (\autoref{tbl:example-sidewaystable}).
\begin{sidewaystable}
    \centering
    \caption[Ejemplo: Rotación de Tablas]{Rotación de Tablas}
    \label{tbl:example-sidewaystable}
    \begin{tabularx}{\columnwidth}{@{}XX@{}}
        \toprule
        \textbf{Column 1} & \textbf{Column 2}\\
        \hline
        \hline
        Second First & Second Second\\
        \blindtext & \blindtext\\
        \bottomrule
    \end{tabularx}
\end{sidewaystable}

\begin{Verbatim}[frame=lines, label=\autoref{tbl:example-sidewaystable} Tabla Rotada
, fontsize=\footnotesize, numbers=left
, baselinestretch=1
, formatcom=\color{gray}]
\begin{sidewaystable}
    \centering
    \caption[Ejemplo: Rotación de Tablas]{Rotación de Tablas}
    \label{tbl:example-sidewaystable-demo}
    \begin{tabularx}{\columnwidth}{@{}XX@{}}
        \toprule
        \textbf{Column 1} & \textbf{Column 2}\\
        \hline
        \hline
        Second First & Second Second\\
        \blindtext & \blindtext\\
        \bottomrule
    \end{tabularx}
\end{sidewaystable}
\end{Verbatim}


\subsection{Opciones Avanzadas para Gráficos}

Los packetes Ti\emph{k}Z y PGF ofrecen alternativas para la creación de gráficos con las más diversas formas y opciones. Para ver opciones consultar \href{http://www.texample.net/tikz/}{www.texample.net/tikz/}.


\newcommand{\MonetaryLevel}{Monetary level}
\newcommand{\RealLevel}{Real level}
\newcommand{\Firms}{Firms}
\newcommand{\Households}{Households}
\newcommand{\Banks}{Banks}
\newcommand{\Commodities}{Commodities}
\newcommand{\LaborPower}{Labor power}
\newcommand{\Wages}{Wages}
\newcommand{\Consumption}{Consumption}
\newcommand{\Credits}{Credits}
\newcommand{\Withdrawals}{Withdrawals}
\newcommand{\Deposits}{Deposits}
\newcommand{\Repayments}{Repayments}

\newcommand{\yslant}{0.5}
\newcommand{\xslant}{-0.6}

\begin{figure}[H]
\centering
\begin{tikzpicture}[scale=1,every node/.style={minimum size=1cm},on grid]

	% Real level
	\begin{scope}[
		yshift=-120,
		every node/.append style={yslant=\yslant,xslant=\xslant},
		yslant=\yslant,xslant=\xslant
	] 
		% The frame:
		\draw[black, dashed, thin] (0,0) rectangle (7,7); 
		% Agents:
		\draw[fill=red]  
			(5,2) circle (.1) % Firms
			(2,2) circle (.1); % Households
		% Flows:
		\draw[-latex,thin] 
			(2,1.8) to[out=-90,in=-90] (5,1.8); % Labour Powers
		\draw[-latex,thin]
			(5,2.2) to[out=90,in=90] (2,2.2); % Wages
		 % Labels:
		\fill[black]
			(0.5,6.5) node[right, scale=.7] {\RealLevel}	
			(5.1,1.9) node[right,scale=.7]{\textbf{\Firms}}
			(1.9,1.9) node[left,scale=.7]{\textbf{\Households}}
			(2.2,3) node [scale=.6, rotate=40] {\Commodities} 
			(4.8,1) node [scale=.6, rotate=40] {\LaborPower};	
	\end{scope}
	
	% 2 vertical lines for linking agents on the 2 levels
	\draw[ultra thin](3.8,4) to (3.8,-0.32);
	\draw[ultra thin](.8,2.4) to (.8,-1.8);
	
	% Monetary level
	\begin{scope}[
		yshift=0,
		every node/.append style={yslant=\yslant,xslant=\xslant},
		yslant=\yslant,xslant=\xslant
	]
		% The frame:
		\fill[white,fill opacity=.75] (0,0) rectangle (7,7); % Opacity
		\draw[black, dashed, thin] (0,0) rectangle (7,7); 
		 % Agents:
		\draw [fill=red]
			(5,2) circle (.1) % Firms
			(2,2) circle (.1) % Households
			(3.5,5) circle (.1); % Banks
		 % Monetary Flows:
		\draw[-latex, thin]
			(3.65,5.1) to[out=30,in=30] (5.15,2.1); % Credits
		\draw[-latex, thin]
			(5,1.8) to[out=-90,in=-90] (2,1.8); % Wages
		\draw[-latex, thin]
			(1.9,2.1) to[out=150,in=150] (3.4,5.1);  % Deposits
		\draw[-latex, thin]
			(3.6,4.9) to[out=-30,in=-30] (2.1,1.9); % Withdrawals
		\draw[-latex, thin]
			(2,2.2) to[out=90,in=90] (5,2.2); % Consumption
		\draw[-latex, thin]
			(4.85,1.9) to[out=210,in=210] (3.35,4.9) ; % Repayments
		 % Labels:
		\fill[black]
			(0.5,6.5) node[right, scale=.7] {\MonetaryLevel}
			(5.1,1.9) node[right,scale=.7]{\textbf {\Firms}}
			(1.9,1.9) node[left,scale=.7]{\textbf {\Households}}
			(3.5,5.1) node[above,scale=.7]{\textbf {\Banks}}
			(5.5,2.8) node [above, scale=.6, rotate=-100] {\Credits}
			(2.6,1.3) node [above, scale=.6, rotate=-10] {\Withdrawals}
			(2.9,4.25) node [above, scale=.6, rotate=50] {\Repayments}
			(2.6,5) node [above, scale=.6, rotate=25] {\Deposits}
			(4.7,2.9) node [above, scale=.6, rotate=-60] {\Consumption}
			(2.3,1.3) node [below, scale=.6, rotate=-40] {\Wages}; 
	\end{scope} 
\end{tikzpicture}
\caption[Gráficos Avanzados con Tikz]{Gráficos Avanzados con Tikz\\ {\scriptsize (Fuente: \url{www.texample.net})}}
\label{fig:tikz}
\end{figure}




\begin{figure}[ht!]
\centering
% Styles
\tikzstyle{load}   = [ultra thick,-latex]
\tikzstyle{stress} = [-latex]
\tikzstyle{dim}    = [latex-latex]
\tikzstyle{axis}   = [-latex,black!55]

% Drawing Views
\tikzstyle{isometric}=[x={(0.710cm,-0.410cm)},y={(0cm,0.820cm)},z={(-0.710cm,-0.410cm)}]
\tikzstyle{dimetric} =[x={(0.935cm,-0.118cm)},y={(0cm,0.943cm)},z={(-0.354cm,-0.312cm)}]
\tikzstyle{dimetric2}=[x={(0.935cm,-0.118cm)},z={(0cm,0.943cm)},y={(+0.354cm,+0.312cm)}]
\tikzstyle{trimetric}=[x={(0.926cm,-0.207cm)},y={(0cm,0.837cm)},z={(-0.378cm,-0.507cm)}]

  \begin{tikzpicture}[scale=.8]
    \node (origin) at (0,0) {}; % shift relative baseline
    \coordinate (O) at (2,3);
    \draw[fill=gray!10] (O) circle (1);
    \draw[fill=white] (O) circle (0.75) node[below,yshift=-1.125cm] {Signpost Cross Section};
    \draw[dim] (O) ++(-0.75,0) -- ++(1.5,0) node[midway,above] {$d_i$};
    \draw[dim] (O) ++(-1,1.25) -- ++(2,0) node[midway,above] {$d_o$}; 
    \foreach \x in {-1,1} {
      \draw (O) ++(\x,0.25) -- ++(0,1.25);
    }
  \end{tikzpicture}%
  \begin{tikzpicture}[dimetric2]
        \coordinate (O) at (0,0,0);
        \draw[axis] (O) -- ++(6,0,0) node[right] {$x$};
        \draw[axis] (O) -- ++(0,6,0) node[above right] {$y$};
        \draw[axis] (O) -- ++(0,0,6) node[above] {$z$};
        \draw[fill=gray!50] (0,0,-0.5) circle (0.5); 
        \fill[fill=gray!50] (-0.46,-0.2,-0.5) -- (0.46,0.2,-0.5) -- (0.46,0.2,0) -- (-0.46,-0.2,0) -- cycle;
        \draw[fill=gray!20] (O) circle (0.5);
    \draw (0.46,0.2,-0.5) -- ++(0,0,0.5) node[below right,pos=0.0] {Fixed Support};
    \draw (-0.46,-0.2,-0.5) -- ++(0,0,0.5);
    \draw[fill=gray!10] (O) circle (0.2);
    \fill[fill=gray!10] (-0.175,-0.1,0) -- (0.175,0.1,0) -- ++(0,0,4) -- (-0.175,-0.1,4) -- cycle;
    \draw (-0.175,-0.1,0) -- ++(0,0,4);
    \draw (0.175,0.1,0) -- ++(0,0,4) node[right,midway] {Steel Post};
    \draw (4,0,3.95) -- ++(0,0,-1);
    \foreach \z in {0.5,0.75,...,5} {
      \draw[-latex] (-2*\z/5-0.2,0,\z) -- (-0.2,0,\z);
    }
    \draw[load] (0,0,4) -- ++(0,0,-1.25) node[right,xshift=0.1cm] {$F_{z1}$};
    \draw[fill=gray!20] (-0.25,-0.25,5) -- (4,-0.25,5) -- (4,+0.25,5) -- (-0.25,+0.25,5) -- cycle; 
    \draw[fill=gray!50] (+4.00,-0.25,4) -- (4,+0.25,4) -- (4,+0.25,5) -- (+4.00,-0.25,5) -- cycle; 
    \draw[fill=gray!10] (-0.25,-0.25,4) -- (4,-0.25,4) -- (4,-0.25,5) -- (-0.25,-0.25,5) -- cycle; 
    \draw (4.05,0,4) -- ++(1,0,0);
    \draw (4.05,0,5) -- ++(1,0,0);
    \draw[dim] (4.5,0,0) -- ++(0,0,4) node[midway,right] {$h_1$};
    \draw[dim] (4.5,0,4) -- ++(0,0,1) node[midway,right] {$h_2$};
    \draw[dim] (0,0,3.4) -- ++(4,0,0) node[midway,below] {$b_2$};
    \coordinate (P) at (2,-0.25,4.5);
    \draw (P) -- ++(0,0,0.25);
    \draw (P) -- ++(0.25,0,0);
    \draw[dim] (2.125,-0.25,4.5) -- ++(0,0,-0.5) node[midway,right] {$z_1$};
    \draw[dim] (2,-0.25,4.625) -- ++(-2,0,0) node[midway,below] {$x_1$};
    \draw[load] (2,-2.45,4.5) -- ++(0,2.2,0) node[pos=0.0,right,xshift=0.08cm] {$F_{y1}$};
    \draw[axis,dashed,-] (O) -- (0,0,5);
    \draw (0,0,5.5) -- ++(4,0,0) node[midway,above] {$w_{z}$};
    \foreach \x in {0,0.25,...,4} {
      \draw[-latex] (\x,0,5.5) -- ++(0,0,-0.5);
    }
    \draw (-0.2,0,0) -- ++(-2,0,5) node[above,xshift=0.5cm] {$w_{x}=\frac{z}{h_1+h_2} w_0$};
  \end{tikzpicture} 
  \caption [Cargas aplicadas sobre un poste.]{Cargas aplicadas sobre un poste.\\ {\scriptsize (Fuente: \url{www.texample.net})}}
\end{figure}




%!TEX root = ../memoria.tex

\chapter{Formatos UTFSM para Memorias y Tesis de Grado }

Los formatos exigidos (y ocupados en este documento) por el Departamento de Industrias y la UTFSM incluyen:

\begin{description}
\item[Tipografía.] Fuente \emph{Times New Roman} o similar de 11 o 12 puntos (pts.), con interlineado de 1 espacio (máximo 1,5 espacios).
\item[Márgenes.] Margen izquierdo (o interno) de $3.5cm$ (mínimo). Margen derecho (o externo) de $2cm$ (mínimo). Note que esto cambia para páginas pares e impares para facilitar el empaste de documentos impresos por ambos lados de cada hoja.
\item[Citas bibliogáficas.] Las citas bibliográficas se harán siguiendo normas de la UTFSM (éstas están basadas en las normas \emph{APA} (usada en este documento), \emph{AMS}, o \emph{IEEE}). Ejemplo:

\begin{quote}
    ``\LaTeX{} es un sistema de diagramación de documentos.'' \citep{Lamport94}.
\end{quote}

Este documento ocupa estas normas. Revisar la bibliografía que se adjunta para ver un ejemplo.

\item[Numeración de Títulos.] El texto del informe final debe ser subdivido en: capítulos y sub-capítulos. La numeración de capítulos estará basada en esquema con división de puntos para los sub-capítulos, es decir: Capítulo 1, Sub-capítulo 1.1, etc.
\item[Numeración de Páginas.] Todas las páginas (con excepción de la portada) deben estar numeradas. El preámbulo (Índices, Resumen, Abstract, etc.) debe llevar numeración distinta del desarrollo (capítulos) del documento.
\item[Numeración de Formulas, Tablas y Figuras.] Las fórmulas, figuras y tablas correspondientes a un mismo capítulo, se identificarán mediante dos números. El primero corresponde al capítulo pertinente y el segundo al número de orden correlativo.


Los números con que se identifican las fórmulas se colocarán al extremo derecho de las mismas y entre paréntesis. Ejemplo (\autoref{eq:eq_example}):
\begin{equation}
f(x) = x^2-2x+1
\label{eq:eq_example}
\end{equation}

Las ilustraciones (gráficos, láminas, fotografías, etc.) en lo posible deben quedar ubicadas dentro de la página que se les referencia. Los números correspondientes a figuras se colocarán en la parte inferior de las mismas, seguidos de título o breve explicación de la figura. Ver \autoref{fig:figure_example}.
	\begin{figure}[ht!]
	\centering
	%\rule{4cm}{3cm}
	\includegraphics[width=.3\textwidth]{figures/logoind.png}
	
	\caption[Logotipo del Departamento de Industrias, UTFSM.]{Logotipo del Departamento de Industrias, UTFSM.\\
	{\footnotesize (Fuente: Departamento de Industrias, 2016.)}}
	
	\label{fig:figure_example}
	\end{figure}

Los números asignados a las tablas se colocarán en la parte superior de ellas, seguidos de los títulos correspondientes. Ver \autoref{tbl:table_example}

\begin{table}[ht]
\centering
\caption{Ejemplo de Numeración de Tablas.}
\begin{tabular}{p{3cm}|p{3cm}|p{3cm}}
\hline
\textbf{Columna 1} & \textbf{Columna 2} & \textbf{Columna 3} \\
\hline\hline
... & ... & ... \\
\hline
... & ... & ... \\
\hline
\end{tabular}
\label{tbl:table_example}
\end{table}

\end{description}

\section{Otros Formatos UTFSM}
\subsection{Formato de las Cubiertas (Empaste)}

La cubierta o tapa será de empaste duro, cubierta de vinilo o similar de color NEGRO con letras doradas, según se muestran en \autoref{fig:thesis_cover} y \autoref{fig:thesis_cover_lateral}.
\begin{figure}[ht!]
\centering
\includegraphics[width=.7\textwidth]{figures/thesis_cover.png}
\caption{Cubierta (Empaste) Memorias y Tesis UTFSM.}
\label{fig:thesis_cover}
\end{figure}

\begin{figure}[ht!]
\centering
\includegraphics[width=.7\textwidth]{figures/thesis_cover_lateral.png}
\caption{Lomo del Empaste para Memorias y Tesis UTFSM.}
\label{fig:thesis_cover_lateral}
\end{figure}

\subsection{Formato del Disco Compacto}

El CD/DVD debe tener una carátula de identificación circular con fondo blanco, conteniendo las siguientes leyendas:

\begin{itemize}
		\item
    Centrado en la parte superior: UTFSM, con letras mayúsculas en negrita tamaño 12. A renglón seguido el nombre de la Unidad Académica con letras mayúsculas en negrita tamaño 10.
		\item
    Centrado en la parte inferior el nombre completo del alumno con letras mayúsculas en negrita tamaño 10.
		\item
    Tres espacios más abajo y centrado, “TÍTULO DE LA MEMORIA”, con letras mayúsculas en negrita tamaño 10.
		\item
    Dos espacios más abajo y centrado MES –AÑO, con letras mayúsculas en negrita tamaño 10.
    En el lado izquierdo y centrado, el escudo en colores de la Institución.
		\item
    En el lado derecho y centrado, NOMBRE DE LA UNIDAD ACADÉMICA y la ubicación CIUDAD – PAIS, con letras mayúsculas en negrita tamaño 8.
\end{itemize}

\begin{figure}[ht!]
\centering
\includegraphics[width=.4\textwidth]{figures/thesis_cd.png}
\caption{Disco Compacto para Memoria UTFSM}
\label{fig:thesis_cd}
\end{figure}


Los CD se guardarán, en la biblioteca, en una caja de acrílico que tendrá una carátula de identificación dividida en tres franjas iguales, con las siguientes leyendas:
\begin{itemize}
		\item
    El escudo a color de la Institución de 20 mm de alto, centrado en la franja superior.
		\item
    El nombre completo del alumno, y centrado dos espacios más abajo el título de la memoria, en la franja del medio
		\item
    El nombre de la Unidad Académica, y renglón más abajo, año. En la franja inferior.
\end{itemize}


\begin{figure}[ht!]
\centering
\includegraphics[width=.4\textwidth]{figures/thesis_cd_cover.png}
\caption{Cubierta de Disco Compacto para Memorias y Tesis UTFSM.}
\label{fig:thesis_cd_cover}
\end{figure}


La carpeta \inlinecode{figures} incluye los diagramas (formato LibreOffice) para modificación e impresión.

%%%%%
\section{Documentos que se incluyen}

Se incluyen (en la carpeta \inlinecode{figures}) logotipos oficiales\footnote{Éstas son imágenes registradas y propiedad intelectual de la UTFSM y del Departamento de Industrias, y no están incluidas en la licencia de esta plantilla. La imagen corporativa de la UTFSM y del Departamento de Industrias están protegidas por leyes chilenas e internacionales de Derechos de autor. Su uso sólo está autorizado a estudiantes y memoristas de la UTFSM para fines de preparación de documentos académicos, incluidas memorias y tesis.}
de la UTFSM y del Departamento de Industrias.

\begin{figure}[ht!]
\centering
\includegraphics[scale = .5]{figures/logousm.png}
\caption{Logotipo de la UTFSM}
\label{fig:logousm}
\end{figure}

\begin{figure}[ht!]
\centering
\includegraphics[scale = .45]{figures/logousmleyenda.png}
\caption{Logotipo de la UTFSM (con leyenda)}
\label{fig:logousmleyenda}
\end{figure}


\begin{figure}[ht!]
\centering
\includegraphics[scale = .75]{figures/logousmind.jpg}
\caption{Logotipo de la UTFSM - Departamento de Industrias}
\label{fig:logousmind}
\end{figure}

\begin{figure}[ht!]
    \centering
    %\rule{4cm}{3cm}
    \includegraphics[width=.8\textwidth]{figures/logo_utfsm_di.png}
    \caption{Logotipo del Departamento de Industrias, UTFSM (Formato lateral).}
    \label{fig:logodiutfsm}
\end{figure}

\begin{figure}[ht!]
\centering
%\rule{4cm}{3cm}
\includegraphics[width=.3\textwidth]{figures/logoind.png}
\caption{Logotipo del Departamento de Industrias, UTFSM }
\label{fig:logoind1}
\end{figure}


%!TEX root = ../memoria.tex

\chapter{\LaTeX}


\section{Obtener \LaTeX{}}

\LaTeX{} es un sistema de preparación de documentos de alta calidad
visual \parencite{latex:whatis}. Si no ha ocupado \LaTeX{} anteriormente,
visite esta página:
\begin{itemize}
\item \href{http://www.latex-project.org/}{http://www.latex-project.org/}
\end{itemize}
\begin{figure}[H]
\begin{centering}
\includegraphics[width=0.7\textwidth]{figures/fig_latex_project}
\par\end{centering}

\caption{LaTeX Project}
\end{figure}


Puede obtener, en forma gratuita, las distribuciones de \LaTeX{},
según su plataforma, en:

\begin{description}
\item [Windows] \href{http://miktex.org/}{http://miktex.org/}; también puede
ocupar \href{http://www.tug.org/protext/}{http://www.tug.org/protext/}.

MikTex ofrece una versión básica. Después de instalarlo, asegúrese de descargar los paquetes adicionales requeridos para compilar esta plantilla.

\item [MacOS] \href{http://www.tug.org/mactex/}{http://www.tug.org/mactex/}.

La versión de MacTex es completa e incluye por defecto todos los paquetes necesarios para compilar esta plantilla.

\item [Unix/Linux] \href{http://www.tug.org/texlive/}{http://www.tug.org/texlive/}.

La instalación de TexLive en plataformas *nix es muy sencilla y directa a través de una consola (con permisos de administración):

(K/X)Ubuntu / Debian: \inlinecode{# apt-get install texlive}

Fedora: \inlinecode{# dnf install texlive}

RedHat / CentOS: \inlinecode{# yum install texlive}
\end{description}

Para una referencia completa sobre \LaTeX{}, recomendamos el libro
de \parencite{Lamport94}; aunque para solucionar problemas específicos,
su mejor aliado es Internet. Otros libros que puede consultar se presentan
en la Bibliografía \parencites{Mittelbach04}{Oetiker06}{Roberts05}{Borbon2014}. %Mittelbach04,Oetiker06,Roberts05,Borbon2014}.


\section{Editores para \LaTeX}
Existen muchos editores de \LaTeX, la mayoría de ellos de distribución gratuita y con versiones para los distintos sistemas operativos:
\begin{description}
    \item [TexStudio] Mac, Windows y Linux. \href{www.texstudio.org}{www.texstudio.org}.
    \item [TexMaker] Mac, Windows y Linux.  \href{https://www.xm1math.net/texmaker/}{https://www.xm1math.net/texmaker/}.
    \item[TeXworks] Mac, Window y Linux. \href{https://www.tug.org/texworks/}{https://www.tug.org/texworks/}
    \item [TexShop] Mac. \href{http://pages.uoregon.edu/koch/texshop/}{http://pages.uoregon.edu/koch/texshop/}.
    \item[Gnome Latex] Linux. \href{https://gitlab.gnome.org/swilmet/gnome-latex}{https://gitlab.gnome.org/swilmet/gnome-latex}.
    \item [Vim + Latex Suite] Mac, Windows y Linux. \href{https://www.vim.org}{https://www.vim.org}.
    \item [LyX] Mac, Windows y Linux. \href{https://www.lyx.org}{https://www.lyx.org}.
    \item [Overleaf] Online. \href{https://www.overleaf.com}{https://www.overleaf.com}.
    \item [Papeeria] Online. \href{https://papeeria.com/landing}{https://papeeria.com/landing}.
    \item [Authorea] Online. \href{https://www.authorea.com}{https://www.authorea.com}.
    \item [\LaTeX{} Base] Online. \href{https://latexbase.com/}{https://latexbase.com/}.
    \item [Latex Workshop for VCode] VCode Plugin. \href{https://marketplace.visualstudio.com/items?itemName=James-Yu.latex-workshop}{https://marketplace.visualstudio.com/}.
\end{description}

%...              % Agregar aquí más capítulos
\end{Verbatim}



\subsection*{\color{red}¡Importante!}

\paragraph{Impresión por ambos lados.}
Este documento está preparado para ser impreso por ambos lados de una hoja (\emph{``twoside''}). Para cambiar esto, en la ``clase de documento'', reemplazar la palabra \emph{``twoside''} por \emph{``oneside''}. Es por esto que encontrará algunas hojas que están en blanco, aparentemente sin motivo.


\begin{Verbatim}[frame=lines, label=\inlinecode{memoria.tex} (extracto)
, fontsize=\footnotesize
, baselinestretch=1
, formatcom=\color{gray}]
%---------------------------------------------------------------------------
%%% DOCUMENT CLASS
\documentclass[
11pt,
letterpaper,
twoside
]{book}
%---------------------------------------------------------------------------
\end{Verbatim}


Es posible que debas cambiar otras configuraciones también para imprimir por un sólo lado. En particular aquellas páginas en blanco después de los agradecimientos y dedicatoria.

Contribuye con el ahorro de papel, no ocupes más hojas de las necesarias.

\paragraph{Codificación de caracteres.}

Todos los archivos \inlinecode{*.tex} de esta plantilla han sido preparados ocupando la codificación de caracteres por defecto \emph{unicode} (UTF-8). Windows (y algunas versiones de OSX) ocupan otro tipo de codificación (ej. \emph{Windows-1252} o \emph{Mac Roman}).

Si deseas ocupar esta plantilla y encuentras problemas con los caracteres acentuados, entonces puedes optar por una de estas tres alternativas:
\begin{enumerate}[i)]
    \item cambiar tu editor (TexMaker, TexStudio, TexShop, etc.) para que ocupe UTF-8 como codificación de caracteres por defecto; o
    \item cambiar la codificación de cada documento \inlinecode{*.tex} para que ocupe la codificación nativa de tu sistema operativo; y, sustituir en el archivo \inlinecode{memoria.tex} la línea (\#62) que dice:
    
    \inlinecode{\\usepackage[utf8x]\{inputenc\}}, por el texto \inlinecode{\\usepackage[latin1]\{inputenc\}}.
    \item escribir todo ocupando caracteres pre-acentuados (ej. \inlinecode{\\'a} en lugar de á).
\end{enumerate}

\begin{framed}
    \textbf{Recuerda:} Mezclar documentos de distintas codificaciones puede generarte muchos problemas al momento de compilar.  
\end{framed}



%%%%%
\section{Diagramación}
Este documento fue realizado usando \LaTeX{} (\citeauthor{latex:whatis}), aunque puede fácilmente ser exportado a LyX (\citeauthor{lyx}). Para ver como transformarlo a Lyx, puede revisar el Wiki (\citeauthor{wikilyx}).

Usted necesitará un compilador de \LaTeX. Los más comúnmente ocupados son \citeauthor{miktex} (Windows) y \citeauthor{mactex} (Apple); Sistemas *nix (incluyendo linux) traen \TeX{} por defecto.

Para una referencia completa sobre \LaTeX{}, recomendamos el libro de \cite{Lamport94}; aunque para solucionar problemas específicos, su mejor aliado es Internet.

% Other Author (Included only in Bibliography)
También puede revisar \citet{Roberts05}, \citet{Oetiker06}, y \citet{Mittelbach04}.


\section{Requisitos}
Los formatos para la generación este documento están contenidos en la Hoja de Estilo \inlinecode{thesis_utfsm.sty}.

Para que funcione correctamente se requiere tener instaladas (como mínimo) las siguientes extensiones \LaTeX{}:
\begin{Verbatim}[frame=lines, label=Paquetes requeridos por \inlinecode{thesis_utfsm.sty}
				, fontsize=\footnotesize
				, baselinestretch=1
				, formatcom=\color{gray}]
geometry    % Márgenes y tamaño de páginas
natbib      % Bibliografía
fontenc     % Codificación de Caracteres
inputenc    % Métodos de entrada (acentos)
fancyhdr    % Encabezados 'Fancy'
chngcntr    % Formatos de Pie de Página
booktabs    % Tablas
tabularx    % Tablas
multirow    % Tablas con multi-columnas / multi-filas
array       % Matrices
float       % Imágenes Flotantes
textcomp    % Símbolos de uso común
endnotes    % Notas finales del documento
paralist    % Mejores Listados
listings    % Mejores Listados
framed      % Marcos
fancybox    % Marcos 'Fancy'
verbatim    % Código Fuente
fancyvrb    % Código Fuente 'Fancy'
wrapfig     % Figuras flotantes
xcolor      % Colores personalizados
graphix     % Mejor inclusión de figuras
subfig      % Figuras con múltiples leyendas
tikz        % Diagramas vectoriales
caption     % Mejores leyendas para figuras y tablas
tocbibind   % Bibliografía en la Tabla de Contenidos
rotating    % Rotación de Tablas
asmmath     % Notación ciéntifica / matemática
asmsymb     % Símbolos matemáticos y letras griegas
txfonts     % Times New Roman (para sistemas distintos de Windows)
microtype   % Mejoras subliminales en el uso de fuentes
parskip     % Separación entre párrafos
\end{Verbatim}

La mayoría de las distribuciones \LaTeX{} traen estos paquetes por defecto, sin embargo, en Windows es posible que deba instalar algunos de ellos si ha instalado el archivo básico de MikTeX.


\section{Figuras}
La siguiente es una figura basada en el archivo \inlinecode{figures/logoind.png}. En este caso, la descripción de la figura va en la parte inferior (ver \autoref{fig:logoind2}).

% Inclusión de Figuras
\begin{figure}[ht!]
\centering
\includegraphics[width=.4\textwidth]{figures/logoind.png}
\caption[Logotipo Departamento de Industrias]{Logotipo Departamento de Industrias\\
{\scriptsize (Fuente: Departamento de Industrias)}}
\label{fig:logoind2}
\end{figure}

La forma de incorporar la \autoref{fig:logoind2} se muestra a continuación:


\begin{Verbatim}[frame=lines, label=Incorporar \autoref{fig:logoind2}
				, fontsize=\footnotesize, numbers=left
				, baselinestretch=1
				, formatcom=\color{gray}]
\begin{figure}[h]
\centering
\includegraphics[width=.4\textwidth]{figures/logoind.png}
\caption[Logotipo Departamento de Industrias]{Logotipo Departamento de Industrias\\
{\scriptsize (Fuente: Departamento de Industrias)}}
\label{fig:logoind2}
\end{figure}
\end{Verbatim}

Otra forma de incorporar figuras es mediante un \inlinecode{float}. En este caso, la figura es incorporada como una imagen ``flotante'' a un costado del texto  (ver Figura \autoref{fig:logousm_float}).

\begin{wrapfigure}{o}{.4\textwidth}
    \vspace{-20pt}
    \begin{spacing}{1}
        \begin{center}
            \includegraphics[width=.35\columnwidth]{figures/logousm.png}
            \vspace{-10pt}
            \caption{Logotipo USM (Float)}
            \label{fig:logousm_float}
        \end{center}
    \end{spacing}
    \vspace{-10pt}
\end{wrapfigure}

Lorem ipsum dolor sit amet, consectetuer adipiscing elit. Ut purus elit, vestibulum ut, placerat ac, adipiscing vitae, felis. Curabitur dictum gravida mauris. Nam arcu libero, nonummy eget, consectetuer id, vulputate a, magna. Donec vehicula augue eu neque. Pellentesque habitant morbi tristique senectus et netus et malesuada fames ac turpis egestas. Mauris ut leo. Cras viverra metus rhoncus sem. Nulla et lectus vestibulum urna fringilla ultrices. Phasellus eu tellus sit amet tortor gravida placerat. Integer sapien est, iaculis in, pretium quis, viverra ac, nunc. Praesent eget sem vel leo ultrices bibendum. Aenean faucibus. Morbi dolor nulla, malesuada eu, pulvinar at, mollis ac, nulla. Curabitur auctor semper nulla. Donec varius orci eget risus. Duis nibh mi, congue eu, accumsan eleifend, sagittis quis, diam. Duis eget orci sit amet orci dignissim rutrum.



\begin{Verbatim}[frame=lines, label=\autoref{fig:logousm_float}
				, fontsize=\footnotesize, numbers=left
				, baselinestretch=1
				, formatcom=\color{gray}]
\begin{wrapfigure}{o}{.4\textwidth}
    \vspace{-20pt}
    \begin{spacing}{1}
        \begin{center}
            \includegraphics[width=.35\columnwidth]{figures/logousm.png}
            \vspace{-10pt}
            \caption{Logotipo USM (Float)}
            \label{fig:logousm_float}
        \end{center}
    \end{spacing}
    \vspace{-10pt}
\end{wrapfigure}
\end{Verbatim}

\newpage

\section{Tablas}

La siguiente es una tabla o cuadro básica (ver \autoref{tbl:temperaturas}). Notar las referencias cruzadas y el título de la tabla en la parte superior.

\begin{table}[h!]
    \caption{Tabla de Temperaturas}\label{tbl:temperaturas}
    \begin{tabularx}{\linewidth}{  l  c  c  X }
    \hline
    \textbf{\textsc{Day}} &  \textbf{\textsc{Min Temp}} 
    		& \textbf{\textsc{Max Temp}} & \textbf{\textsc{Summary}}\\
	  \hline\hline
    Monday & 11C & 22C & A clear day with lots of sunshine.
    However, the strong breeze will bring down the temperatures. \\ \hline
    Tuesday & 9C & 19C & Cloudy with rain, across many northern regions. Clear spells
    across most of Scotland and Northern Ireland,
    but rain reaching the far northwest. \\ \hline
    Wednesday & 10C & 21C & Rain will still linger for the morning.
    Conditions will improve by early afternoon and continue
    throughout the evening. \\
    \hline
    \end{tabularx}
\end{table}

\begin{Verbatim}[frame=lines, label=\autoref{fig:logousm_float}
				, fontsize=\footnotesize, numbers=left
				, baselinestretch=1
				, formatcom=\color{gray}]
\begin{table}[h!]
    \caption{Tabla de Temperaturas}\label{tbl:temperaturas}
    \begin{tabularx}{\linewidth}{  l  c  c  X }
    \hline
    \textbf{\textsc{Day}} &  \textbf{\textsc{Min Temp}} 
    		& \textbf{\textsc{Max Temp}} & \textbf{\textsc{Summary}}\\
	  \hline\hline
    Monday & 11C & 22C & A clear day with lots of sunshine.
    However, the strong breeze will bring down the temperatures. \\ \hline
    Tuesday & 9C & 19C & Cloudy with rain, across many northern regions. Clear spells
    across most of Scotland and Northern Ireland,
    but rain reaching the far northwest. \\ \hline
    Wednesday & 10C & 21C & Rain will still linger for the morning.
    Conditions will improve by early afternoon and continue
    throughout the evening. \\
    \hline
    \end{tabularx}
\end{table}
\end{Verbatim}



\section{Rotación de Tablas}
En caso de tener tablas muy grandes, o si necesita una tabla rotada.
\begin{sidewaystable}
    \centering
    \caption{Rotación de Tablas}
    \begin{tabularx}{\columnwidth}{X X}
        \hline\hline
        \textbf{Column 1} & \textbf{Column 2}\\
        \hline
        Second First & Second Second\\
        \blindtext & \blindtext\\
        \hline\hline
    \end{tabularx}
\end{sidewaystable}


\newpage

\section{Opciones Avanzadas para Gráficos}

Los packetes Ti\emph{k}Z y PGF ofrecen alternativas para la creación de gráficos con las más diversas formas y opciones. Para ver opciones consultar \href{http://www.texample.net/tikz/}{www.texample.net/tikz/}.


\newcommand{\MonetaryLevel}{Monetary level}
\newcommand{\RealLevel}{Real level}
\newcommand{\Firms}{Firms}
\newcommand{\Households}{Households}
\newcommand{\Banks}{Banks}
\newcommand{\Commodities}{Commodities}
\newcommand{\LaborPower}{Labor power}
\newcommand{\Wages}{Wages}
\newcommand{\Consumption}{Consumption}
\newcommand{\Credits}{Credits}
\newcommand{\Withdrawals}{Withdrawals}
\newcommand{\Deposits}{Deposits}
\newcommand{\Repayments}{Repayments}

\newcommand{\yslant}{0.5}
\newcommand{\xslant}{-0.6}

\begin{figure}[H]
\centering
\begin{tikzpicture}[scale=1,every node/.style={minimum size=1cm},on grid]

	% Real level
	\begin{scope}[
		yshift=-120,
		every node/.append style={yslant=\yslant,xslant=\xslant},
		yslant=\yslant,xslant=\xslant
	] 
		% The frame:
		\draw[black, dashed, thin] (0,0) rectangle (7,7); 
		% Agents:
		\draw[fill=red]  
			(5,2) circle (.1) % Firms
			(2,2) circle (.1); % Households
		% Flows:
		\draw[-latex,thin] 
			(2,1.8) to[out=-90,in=-90] (5,1.8); % Labour Powers
		\draw[-latex,thin]
			(5,2.2) to[out=90,in=90] (2,2.2); % Wages
		 % Labels:
		\fill[black]
			(0.5,6.5) node[right, scale=.7] {\RealLevel}	
			(5.1,1.9) node[right,scale=.7]{\textbf{\Firms}}
			(1.9,1.9) node[left,scale=.7]{\textbf{\Households}}
			(2.2,3) node [scale=.6, rotate=40] {\Commodities} 
			(4.8,1) node [scale=.6, rotate=40] {\LaborPower};	
	\end{scope}
	
	% 2 vertical lines for linking agents on the 2 levels
	\draw[ultra thin](3.8,4) to (3.8,-0.32);
	\draw[ultra thin](.8,2.4) to (.8,-1.8);
	
	% Monetary level
	\begin{scope}[
		yshift=0,
		every node/.append style={yslant=\yslant,xslant=\xslant},
		yslant=\yslant,xslant=\xslant
	]
		% The frame:
		\fill[white,fill opacity=.75] (0,0) rectangle (7,7); % Opacity
		\draw[black, dashed, thin] (0,0) rectangle (7,7); 
		 % Agents:
		\draw [fill=red]
			(5,2) circle (.1) % Firms
			(2,2) circle (.1) % Households
			(3.5,5) circle (.1); % Banks
		 % Monetary Flows:
		\draw[-latex, thin]
			(3.65,5.1) to[out=30,in=30] (5.15,2.1); % Credits
		\draw[-latex, thin]
			(5,1.8) to[out=-90,in=-90] (2,1.8); % Wages
		\draw[-latex, thin]
			(1.9,2.1) to[out=150,in=150] (3.4,5.1);  % Deposits
		\draw[-latex, thin]
			(3.6,4.9) to[out=-30,in=-30] (2.1,1.9); % Withdrawals
		\draw[-latex, thin]
			(2,2.2) to[out=90,in=90] (5,2.2); % Consumption
		\draw[-latex, thin]
			(4.85,1.9) to[out=210,in=210] (3.35,4.9) ; % Repayments
		 % Labels:
		\fill[black]
			(0.5,6.5) node[right, scale=.7] {\MonetaryLevel}
			(5.1,1.9) node[right,scale=.7]{\textbf {\Firms}}
			(1.9,1.9) node[left,scale=.7]{\textbf {\Households}}
			(3.5,5.1) node[above,scale=.7]{\textbf {\Banks}}
			(5.5,2.8) node [above, scale=.6, rotate=-100] {\Credits}
			(2.6,1.3) node [above, scale=.6, rotate=-10] {\Withdrawals}
			(2.9,4.25) node [above, scale=.6, rotate=50] {\Repayments}
			(2.6,5) node [above, scale=.6, rotate=25] {\Deposits}
			(4.7,2.9) node [above, scale=.6, rotate=-60] {\Consumption}
			(2.3,1.3) node [below, scale=.6, rotate=-40] {\Wages}; 
	\end{scope} 
\end{tikzpicture}
\caption[Gráficos Avanzados con Tikz]{Gráficos Avanzados con Tikz\\ {\scriptsize (Fuente: \url{www.texample.net})}}
\label{fig:tikz}
\end{figure}


\begin{figure}[ht!]
\centering
\usetikzlibrary{chains,fit,shapes}
\begin{tikzpicture}
\tikzstyle{every path}=[very thick]

\edef\sizetape{0.7cm}
\tikzstyle{tmtape}=[draw,minimum size=\sizetape]
\tikzstyle{tmhead}=[arrow box,draw,minimum size=.5cm,arrow box
arrows={east:.25cm, west:0.25cm}]

%% Draw TM tape
\begin{scope}[start chain=1 going right,node distance=-0.15mm]
    \node [on chain=1,tmtape,draw=none] {$\ldots$};
    \node [on chain=1,tmtape] {};
    \node [on chain=1,tmtape] (input) {b};
    \node [on chain=1,tmtape] {b};
    \node [on chain=1,tmtape] {a};
    \node [on chain=1,tmtape] {a};
    \node [on chain=1,tmtape] {a};
    \node [on chain=1,tmtape] {a};
    \node [on chain=1,tmtape] {};
    \node [on chain=1,tmtape,draw=none] {$\ldots$};
    \node [on chain=1] {\textbf{Input/Output Tape}};
\end{scope}

%% Draw TM Finite Control
\begin{scope}
[shift={(3cm,-5cm)},start chain=circle placed {at=(-\tikzchaincount*60:1.5)}]
\foreach \i in {q_0,q_1,q_2,q_3,\ddots,q_n}
	\node [on chain] {$\i$};

% Arrow to current state
\node (center) {};
\draw[->] (center) -- (circle-2);

\node[rounded corners,draw=black,thick,fit=(circle-1) (circle-2) (circle-3) 
      (circle-4) (circle-5) (circle-6),
			label=below:\textbf{Finite Control}] (fsbox)
		{};
\end{scope}

%% Draw TM head below (input) tape cell
\node [tmhead,yshift=-.3cm] at (input.south) (head) {$q_1$};

%% Link Finite Control with Head
\path[->,draw] (fsbox.north) .. controls (4.5,-1) and (0,-2) .. node[right] 
			(headlinetext)
 			{} 
			(head.south);
\node[xshift=3cm] at (headlinetext)  
			{\begin{tabular}{c} 
				\textbf{Reading and Writing Head} \\  
				\textbf{(moves in both directions)} 
			 \end{tabular}};

\end{tikzpicture}
\caption [Diagrama de la Máquina de Türing]{Diagrama de la Máquina de Türing\\ {\scriptsize (Fuente: \url{www.texample.net})}}
\end{figure}


\begin{figure}[ht!]
\centering
% Styles
\tikzstyle{load}   = [ultra thick,-latex]
\tikzstyle{stress} = [-latex]
\tikzstyle{dim}    = [latex-latex]
\tikzstyle{axis}   = [-latex,black!55]

% Drawing Views
\tikzstyle{isometric}=[x={(0.710cm,-0.410cm)},y={(0cm,0.820cm)},z={(-0.710cm,-0.410cm)}]
\tikzstyle{dimetric} =[x={(0.935cm,-0.118cm)},y={(0cm,0.943cm)},z={(-0.354cm,-0.312cm)}]
\tikzstyle{dimetric2}=[x={(0.935cm,-0.118cm)},z={(0cm,0.943cm)},y={(+0.354cm,+0.312cm)}]
\tikzstyle{trimetric}=[x={(0.926cm,-0.207cm)},y={(0cm,0.837cm)},z={(-0.378cm,-0.507cm)}]

  \begin{tikzpicture}[scale=.8]
    \node (origin) at (0,0) {}; % shift relative baseline
    \coordinate (O) at (2,3);
    \draw[fill=gray!10] (O) circle (1);
    \draw[fill=white] (O) circle (0.75) node[below,yshift=-1.125cm] {Signpost Cross Section};
    \draw[dim] (O) ++(-0.75,0) -- ++(1.5,0) node[midway,above] {$d_i$};
    \draw[dim] (O) ++(-1,1.25) -- ++(2,0) node[midway,above] {$d_o$}; 
    \foreach \x in {-1,1} {
      \draw (O) ++(\x,0.25) -- ++(0,1.25);
    }
  \end{tikzpicture}%
  \begin{tikzpicture}[dimetric2]
        \coordinate (O) at (0,0,0);
        \draw[axis] (O) -- ++(6,0,0) node[right] {$x$};
        \draw[axis] (O) -- ++(0,6,0) node[above right] {$y$};
        \draw[axis] (O) -- ++(0,0,6) node[above] {$z$};
        \draw[fill=gray!50] (0,0,-0.5) circle (0.5); 
        \fill[fill=gray!50] (-0.46,-0.2,-0.5) -- (0.46,0.2,-0.5) -- (0.46,0.2,0) -- (-0.46,-0.2,0) -- cycle;
        \draw[fill=gray!20] (O) circle (0.5);
    \draw (0.46,0.2,-0.5) -- ++(0,0,0.5) node[below right,pos=0.0] {Fixed Support};
    \draw (-0.46,-0.2,-0.5) -- ++(0,0,0.5);
    \draw[fill=gray!10] (O) circle (0.2);
    \fill[fill=gray!10] (-0.175,-0.1,0) -- (0.175,0.1,0) -- ++(0,0,4) -- (-0.175,-0.1,4) -- cycle;
    \draw (-0.175,-0.1,0) -- ++(0,0,4);
    \draw (0.175,0.1,0) -- ++(0,0,4) node[right,midway] {Steel Post};
    \draw (4,0,3.95) -- ++(0,0,-1);
    \foreach \z in {0.5,0.75,...,5} {
      \draw[-latex] (-2*\z/5-0.2,0,\z) -- (-0.2,0,\z);
    }
    \draw[load] (0,0,4) -- ++(0,0,-1.25) node[right,xshift=0.1cm] {$F_{z1}$};
    \draw[fill=gray!20] (-0.25,-0.25,5) -- (4,-0.25,5) -- (4,+0.25,5) -- (-0.25,+0.25,5) -- cycle; 
    \draw[fill=gray!50] (+4.00,-0.25,4) -- (4,+0.25,4) -- (4,+0.25,5) -- (+4.00,-0.25,5) -- cycle; 
    \draw[fill=gray!10] (-0.25,-0.25,4) -- (4,-0.25,4) -- (4,-0.25,5) -- (-0.25,-0.25,5) -- cycle; 
    \draw (4.05,0,4) -- ++(1,0,0);
    \draw (4.05,0,5) -- ++(1,0,0);
    \draw[dim] (4.5,0,0) -- ++(0,0,4) node[midway,right] {$h_1$};
    \draw[dim] (4.5,0,4) -- ++(0,0,1) node[midway,right] {$h_2$};
    \draw[dim] (0,0,3.4) -- ++(4,0,0) node[midway,below] {$b_2$};
    \coordinate (P) at (2,-0.25,4.5);
    \draw (P) -- ++(0,0,0.25);
    \draw (P) -- ++(0.25,0,0);
    \draw[dim] (2.125,-0.25,4.5) -- ++(0,0,-0.5) node[midway,right] {$z_1$};
    \draw[dim] (2,-0.25,4.625) -- ++(-2,0,0) node[midway,below] {$x_1$};
    \draw[load] (2,-2.45,4.5) -- ++(0,2.2,0) node[pos=0.0,right,xshift=0.08cm] {$F_{y1}$};
    \draw[axis,dashed,-] (O) -- (0,0,5);
    \draw (0,0,5.5) -- ++(4,0,0) node[midway,above] {$w_{z}$};
    \foreach \x in {0,0.25,...,4} {
      \draw[-latex] (\x,0,5.5) -- ++(0,0,-0.5);
    }
    \draw (-0.2,0,0) -- ++(-2,0,5) node[above,xshift=0.5cm] {$w_{x}=\frac{z}{h_1+h_2} w_0$};
  \end{tikzpicture} 
  \caption [Cargas aplicadas sobre un poste.]{Cargas aplicadas sobre un poste.\\ {\scriptsize (Fuente: \url{www.texample.net})}}
\end{figure}



