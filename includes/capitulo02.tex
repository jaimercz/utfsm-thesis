%!TEX root = ../memoria.tex

\chapter{Formatos UTFSM para Memorias y Tesis de Grado }

Los formatos exigidos (y ocupados en este documento) por el Departamento de Industrias y la UTFSM incluyen\footnote{La imágenes son propiedad de la UTFSM.}:

\begin{description}
\item[Tipografía.] Fuente \emph{Times New Roman} o similar de 11 o 12 puntos (pts.), con interlineado de 1 espacio (máximo 1,5 espacios).
\item[Márgenes.] Margen izquierdo (o interno) de $3.5cm$ (mínimo). Margen derecho (o externo) de $2cm$ (mínimo). Note que esto cambia para páginas pares e impares para facilitar el empaste de documentos impresos por ambos lados de cada hoja.
\item[Citas bibliogáficas.] Las citas bibliográficas se harán siguiendo normas de la UTFSM (éstas están basadas en las normas \emph{APA} (usada en este documento), \emph{AMS}, o \emph{IEEE}). Ejemplo:

\begin{quote}
    ``\LaTeX{} es un sistema de diagramación de documentos.'' \citep{Lamport94}.
\end{quote}

Este documento ocupa estas normas. Revisar la bibliografía que se adjunta para ver un ejemplo.

\item[Numeración de Títulos.] El texto del informe final debe ser subdivido en: capítulos y sub-capítulos. La numeración de capítulos estará basada en esquema con división de puntos para los sub-capítulos, es decir: Capítulo 1, Sub-capítulo 1.1, etc.
\item[Numeración de Páginas.] Todas las páginas (con excepción de la portada) deben estar numeradas. El preámbulo (Índices, Resumen, Abstract, etc.) debe llevar numeración distinta del desarrollo (capítulos) del documento.
\item[Numeración de Formulas] Las fórmulas, figuras y tablas correspondientes a un mismo capítulo, se identificarán mediante dos números. El primero corresponde al capítulo pertinente y el segundo al número de orden correlativo.
    
    Los números con que se identifican las fórmulas se colocarán al extremo derecho de las mismas y entre paréntesis. Ejemplos (\autoref{eq:eq_example}, \autoref{eq:align_example}):
    \begin{equation}
    f(x) = x^2-2x+1
    \label{eq:eq_example}
    \end{equation}

    \begin{align}
    S(\omega) 
    &= \frac{\alpha g^2}{\omega^5} e^{[ -0.74\bigl\{\frac{\omega U_\omega 19.5}{g}\bigr\}^{\!-4}\,]} \\
    \label{eq:align_example}
    &= \frac{\alpha g^2}{\omega^5} \exp\Bigl[ -0.74\Bigl\{\frac{\omega U_\omega 19.5}{g}\Bigr\}^{\!-4}\,\Bigr] 
    \end{align}

\item[Numeración de Figuras.] Las figuras (gráficos) correspondientes a un mismo capítulo, se identificarán mediante dos números. El primero corresponde al capítulo pertinente y el segundo al número de orden correlativo.


Las ilustraciones (gráficos, láminas, fotografías, etc.) en lo posible deben quedar ubicadas dentro de la página que se les referencia.

Los números correspondientes a figuras se colocarán en la parte inferior de las mismas, seguidos de título o breve explicación de la figura. Ver \autoref{fig:figure_example}.
	\begin{figure}[ht!]
	\centering
	%\rule{4cm}{3cm}
	\includegraphics[width=.25\textwidth]{figures/escudo-utfsm.png}
	
	\caption[Escudo oficial de la UTFSM.]{Escudo oficial de la UTFSM.\\
    {\footnotesize (Fuente: UTFSM, 2023.)}}
	
	\label{fig:figure_example}
	\end{figure}

\item[Numeración de Tablas] Las tablas correspondientes a un mismo capítulo, se identificarán mediante dos números. El primero corresponde al capítulo pertinente y el segundo al número de orden correlativo.
Los números asignados a las tablas se colocarán en la parte superior de ellas, seguidos de los títulos correspondientes. Ver \autoref{tbl:table_example}

\begin{table}[ht]
    \centering
    \caption[Ejemplo: Numeración de Tablas]{Ejemplo de Numeración de Tablas.}
    \begin{tabular}{@{}p{3cm}|p{3cm}|p{3cm}@{}}
        \toprule
        \textbf{Columna 1} & \textbf{Columna 2} & \textbf{Columna 3} \\
        \hline\hline
        ... & ... & ... \\
        \hline
        ... & ... & ... \\
        \hline
        ... & ... & ... \\
        \bottomrule
    \end{tabular}
    \label{tbl:table_example}
\end{table}

En caso de tener tablas muy grandes, o si necesita una tabla rotada, puedes ocupar \inlinecode{sidewaystable} (Ejemplo: \autoref{tbl:example-sidewaystable}).

\begin{sidewaystable}
    \centering
    \caption[Ejemplo: Rotación de Tablas]{Rotación de Tablas}
    \label{tbl:example-sidewaystable}
    \begin{tabularx}{\columnwidth}{@{}XX@{}}
        \toprule
        \textbf{Column 1} & \textbf{Column 2}\\
        \hline
        \hline
        Second First & Second Second\\
        \blindtext & \blindtext\\
        \bottomrule
    \end{tabularx}
\end{sidewaystable}

\end{description}

%%%%%
\section{Documentos que se incluyen}

Se incluyen (en la carpeta \inlinecode{figures}) logotipos oficiales\footnote{Éstas son imágenes registradas y propiedad intelectual de la UTFSM y del Departamento de Industrias, y no están incluidas en la licencia de esta plantilla. La imagen corporativa de la UTFSM y del Departamento de Industrias están protegidas por leyes chilenas e internacionales de derechos de autor. Su uso sólo está autorizado a estudiantes y memoristas de la UTFSM para fines de preparación de documentos académicos, incluidas memorias y tesis.}
de la UTFSM y del Departamento de Industrias.

    \begin{figure}[ht!]
        \begin{subfigure}[b]{0.4\textwidth}
        \centering
        \includegraphics[width = .7\textwidth]{figures/escudo-utfsm.png}
        \caption{Escudo de la UTFSM}
        \label{fig:escudo}
        \end{subfigure}%
        \hfill
        \begin{subfigure}[b]{0.4\textwidth}
        \centering
        \includegraphics[width = .6\textwidth]{figures/logoind.png}
        \caption{Logotipo del Departamento de Industrias, UTFSM }
        \label{fig:logoindustrias}
        \end{subfigure}%
        \\
        \bigskip
        \\
        \begin{subfigure}[b]{0.4\textwidth}
        \centering
        \includegraphics[width = .9\textwidth]{figures/logousmleyenda}
        \caption{Logotipo de la UTFSM (con leyenda)}
        \label{fig:logousm_leyenda}
        \end{subfigure}%
        \hfill
        \begin{subfigure}[b]{0.4\textwidth}
        \centering
        \includegraphics[width = .8\textwidth]{figures/logousmind.jpg}
        \caption{Logotipo de la UTFSM - Departamento de Industrias}
        \label{fig:logousm_industrias}
        \end{subfigure}%
        \\
        \bigskip
        \\
        \begin{subfigure}[b]{0.4\textwidth}
        \centering
        \includegraphics[width = .9\textwidth]{figures/logousm-lateral}
        \caption{Logotipo de la UTFSM (con leyenda lateral)}
        \label{fig:logousm_leyenda_lateral}
        \end{subfigure}%
        \hfill
        \begin{subfigure}[b]{0.4\textwidth}
        \centering
        \includegraphics[width=.9\textwidth]{figures/logo_utfsm_di.png}
        \caption{Logotipo del Departamento de Industrias, UTFSM (Formato lateral).}
        \label{fig:logousm_industrias_lateral}
        \end{subfigure}%
        \caption {Archivos (imágenes) Incluidos}
    \end{figure}
