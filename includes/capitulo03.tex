%!TEX root = ../memoria.tex

\chapter{\LaTeX}


\section{Obtener \LaTeX{}}

\LaTeX{} es un sistema de preparación de documentos de alta calidad
visual \parencite{latex:whatis}. Si no ha ocupado \LaTeX{} anteriormente,
visite esta página:
\begin{itemize}
\item \href{http://www.latex-project.org/}{http://www.latex-project.org/}
\end{itemize}
\begin{figure}[H]
\begin{centering}
\includegraphics[width=0.7\textwidth]{figures/fig_latex_project}
\par\end{centering}

\caption{LaTeX Project}
\end{figure}


Puede obtener, en forma gratuita, las distribuciones de \LaTeX{},
según su plataforma, en:

\begin{description}
\item [Windows] \href{http://miktex.org/}{http://miktex.org/}; también puede
ocupar \href{http://www.tug.org/protext/}{http://www.tug.org/protext/}.

MikTex ofrece una versión básica. Después de instalarlo, asegúrese de descargar los paquetes adicionales requeridos para compilar esta plantilla.

\item [MacOS] \href{http://www.tug.org/mactex/}{http://www.tug.org/mactex/}.

La versión de MacTex es completa e incluye por defecto todos los paquetes necesarios para compilar esta plantilla.

\item [Unix/Linux] \href{http://www.tug.org/texlive/}{http://www.tug.org/texlive/}.

La instalación de TexLive en plataformas *nix es muy sencilla y directa a través de una consola (con permisos de administración):

(K/X)Ubuntu / Debian: \inlinecode{# apt-get install texlive}

Fedora: \inlinecode{# dnf install texlive}

RedHat / CentOS: \inlinecode{# yum install texlive}
\end{description}

Para una referencia completa sobre \LaTeX{}, recomendamos el libro
de \parencite{Lamport94}; aunque para solucionar problemas específicos,
su mejor aliado es Internet. Otros libros que puede consultar se presentan
en la Bibliografía \parencites{Mittelbach04}{Oetiker06}{Roberts05}{Borbon2014}. %Mittelbach04,Oetiker06,Roberts05,Borbon2014}.


\section{Editores para \LaTeX}
Existen muchos editores de \LaTeX, la mayoría de ellos de distribución gratuita y con versiones para los distintos sistemas operativos:
\begin{description}
    \item [TexStudio] Mac, Windows y Linux. \href{www.texstudio.org}{www.texstudio.org}.
    \item [TexMaker] Mac, Windows y Linux.  \href{https://www.xm1math.net/texmaker/}{https://www.xm1math.net/texmaker/}.
    \item[TeXworks] Mac, Window y Linux. \href{https://www.tug.org/texworks/}{https://www.tug.org/texworks/}
    \item [TexShop] Mac. \href{http://pages.uoregon.edu/koch/texshop/}{http://pages.uoregon.edu/koch/texshop/}.
    \item[Gnome Latex] Linux. \href{https://gitlab.gnome.org/swilmet/gnome-latex}{https://gitlab.gnome.org/swilmet/gnome-latex}.
    \item [Vim + Latex Suite] Mac, Windows y Linux. \href{https://www.vim.org}{https://www.vim.org}.
    \item [LyX] Mac, Windows y Linux. \href{https://www.lyx.org}{https://www.lyx.org}.
    \item [Overleaf] Online. \href{https://www.overleaf.com}{https://www.overleaf.com}.
    \item [Papeeria] Online. \href{https://papeeria.com/landing}{https://papeeria.com/landing}.
    \item [Authorea] Online. \href{https://www.authorea.com}{https://www.authorea.com}.
    \item [\LaTeX{} Base] Online. \href{https://latexbase.com/}{https://latexbase.com/}.
    \item [Latex Workshop for VCode] VCode Plugin. \href{https://marketplace.visualstudio.com/items?itemName=James-Yu.latex-workshop}{https://marketplace.visualstudio.com/}.
\end{description}
